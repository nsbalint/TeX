\documentclass{article}
\usepackage{t1enc}
\usepackage[magyar]{babel}
\usepackage[utf8]{inputenc}
\usepackage{xcolor}
\usepackage{graphicx}
\usepackage{hulipsum}

\begin{document}
\large \textbf{\textit{Helló Világ! Így néz ki bold és italic-kal.}}
\color{blue}

\texttt{Ez a LateX. Így pedig a typewriter font-tal.}

\color{black}
A szöveg alapból így néz ki de {\sl így is nézhet ki slanteddal.}\bigskip

Tükrözve vízszintesen.\par
\scalebox{-2.0}{\rotatebox[origin=c]{90}{\raisebox{\depth}{\scalebox{1}[-1]{Tükrözve vízszintesen}}}} \par\bigskip
\colorbox{red}{Tükrözve vízszintesen.\par
\rotatebox[origin=c]{270}{\raisebox{\depth}{\scalebox{1}[-1]{Tükrözve vízszintesen}}}} \par\bigskip

Tükrözve függőlegesen.\par
\framebox{\raisebox{\depth}{\scalebox{-1}[-1]{Tükrözve függőlegesen.}}} \par\bigskip
\colorbox{black} {\color{white} Fehér}

\color{red}
\framebox{Piros}
\end{document}