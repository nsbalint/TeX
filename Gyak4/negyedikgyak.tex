\documentclass{article}
\usepackage{amsthm}
\usepackage{amsmath}
\usepackage{mathtools}
\usepackage{amsfonts}
\usepackage{hyperref}
\usepackage{lipsum}
\newtheorem{tet}{Tetel}
\newtheorem{lemma}[tet]{Lemma}
\newtheorem{defin}[tet]{Definíció}
\usepackage{babel}[magyar]

\begin{document}
\section{Elso section}
\begin{tet}[PatakBro]
\lipsum[1]
\end{tet}
\begin{proof}
Sheesh
\end{proof}
\begin{proof}
Big Bro
\end{proof}
\begin{tet}
\lipsum[1]
\end{tet}

\section{Masodik section}

\begin{lemma}
\lipsum[1]
\end{lemma}

\begin{defin}
\lipsum[1]
\end{defin}

\section{Bevezető}

a) Az $\frac{1}{n^2}$ sorösszege:
\[ \sum_{n=1}^\infty \frac{1}{n^2}
= \frac{\pi^2}{6} \]

b)Az n! (n faktoriális) a számok szorzata 1-től n-ig, azaz
\[ n!:=\prod_{k=1}^\infty = 1\cdot 2 \cdot....\cdot n  \]
c) Legyen 0  k  n. A binomiális együttható
\[\binom{n}{k}:= \frac{n!}{k!\cdot(n-k)!}\]

d) Az előjel- azaz szignum függvényt a következőképpen definiáljuk:

\[ sgn(x):=\begin{cases}
1, & \text{ha } x > 0, \\
0, & \text{ha } x = 0 \\ 
-1 & \text{ha } x < 0. \end{cases} \]


\section{Determináns}
\subsection{a}
Legyen\[[n]:=\{1, 2, \ldots, n\}\] a természetes számok halmaza 1-től n-ig.
 
\subsection{b}
Egy \(n\)-edrendű permutáció, \(\sigma\), egy bijekció a \([n]\) halmazból a \([n]\) halmazba. Az \(n\)-edrendű permutációk halmazát, az ún. szimmetrikus csoportot, \(S_n\)-nel jelöljük
 
\subsection{c}
Egy \(\sigma \in S_n\) permutációban inverzió egy \((i, j)\) párt jelent, ha \(i < j\) és \(\sigma_i > \sigma_j\).
 
\subsection{d}
Egy \(\sigma \in S_n\) permutáció paritását az inverziók számával jellemezzük:
\[ I(\sigma) = {|(i, j) \ i < j,\ i, j \in [n],\ \sigma_i > \sigma_j}| \]
 
\subsection{e}
Legyen \(A \in \mathbb{R}^{n \times n}\) egy \(n \times n\)-es (négyzetes) valós mátrix.
\[ A =
\begin{bmatrix}
a_{11} & a_{12} & \cdots & a_{1n} \\
a_{21} & a_{22} & \cdots & a_{2n} \\
\vdots  & \vdots  & \ddots & \vdots  \\
a_{n1} & a_{n2} & \cdots & a_{nn} \\
\end{bmatrix}
\]
Az A mátrix determinánsát a következőképpen definiáljuk:
\[
\det(A) = \begin{bmatrix}
a_{11} & a_{12} & \cdots & a_{1n} \\
a_{21} & a_{22} & \cdots & a_{2n} \\
\vdots  & \vdots  & \ddots & \vdots  \\
a_{n1} & a_{n2} & \cdots & a_{nn} \\
\end{bmatrix} := \sum_{\sigma \in S_n} (-1)^{I(\sigma)} \prod_{i=1}^{n} a_{i\sigma(i)}
\]
 
\section{Logikai azonosság}
\subsection{A)}
\begin{align*}
(a \land b \land c) \rightarrow d & =  a \rightarrow (b \rightarrow (c \rightarrow d)) \\
\end{align*}
 
\subsection{B)}
\begin{align*}
x \rightarrow y & = \overline{x} \lor y  \\
\overline{x \lor y} & = \overline{x} \land \overline{y} \\
\overline{x \land y} & = \overline{x} \lor \overline{y} \\
\end{align*}
 \subsection{C)}
\begin{align*}
(a \land b \land c) \rightarrow d & = \overline{a \land b \land c} \lor d & \\
=(\overline{a} \lor \overline{b} \lor \overline{c}) \lor d  \\
\end{align*}

\section{Binominális képlet}
\begin{subequations}
\begin{align}
(a+b)^{n+1} &= (a+b) \cdot \left( \sum_{k=0}^n \binom{n}{k} a^{n-k}b^k \right)\\
&\nonumber =\cdots\\
&= \sum_{k=0}^n \binom{n}{k} a^{(n+1)-k}b^k + \sum_{k=1}^{n+1} \binom{n}{k-1} a^{(n+1)-k}b^{k} \\
&\nonumber =\cdots\\
\begin{split} &= \binom{n+1}{0} a^{n+1-0} b^0 + \sum_{k=1}^n \binom{n+1}{k} a^{(n+1)-k}b^k \\
&+ \binom{n+1}{n+1} a^{n+1-(n+1)} b^{n+1} \end{split} \\
&= \sum_{k=0}^{n+1} \binom{n+1}{k} a^{(n+1)-k}b^k
\end{align}
\end{subequations}
\end{document}